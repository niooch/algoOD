\documentclass[11pt,a4paper]{article}

\usepackage[T1]{fontenc}
\usepackage[utf8]{inputenc}
\usepackage[polish]{babel}
\usepackage{lmodern}
\usepackage[final]{microtype}
\usepackage{tikz}
\usetikzlibrary{arrows.meta,positioning,calc}

% ======= Parametry do łatwej zmiany liczby węzłów =======
\newcommand{\nI}{3} % liczba dostawców |I|
\newcommand{\nJ}{4} % liczba odbiorców |J|

% ======= Style rysunku =======
\tikzset{
  supplier/.style = {circle, draw, thick, minimum size=8mm, font=\small, fill=blue!5},
  demand/.style   = {circle, draw, thick, minimum size=8mm, font=\small, fill=green!5},
  cost/.style     = {font=\scriptsize, midway, sloped, fill=white, inner sep=1pt},
  supply/.style   = {font=\scriptsize, right=2mm},
  demandlab/.style= {font=\scriptsize, left=2mm},
  every edge quotes/.style = {auto, text=black},
  edge/.style     = {thin, -{Latex[length=2mm]}, draw=black!60}
}


\usepackage[a4paper,margin=2.5cm]{geometry}
\usepackage{fancyhdr}
\pagestyle{fancy}
\fancyhf{} % czyść wszystko
\lhead{Lista 2.}
\rhead{\leftmark}
\cfoot{\thepage}

\usepackage{amsmath,amssymb,amsthm,mathtools}
\usepackage{siunitx} 
\sisetup{locale = PL}
\usepackage{bm}     

\usepackage[hidelinks]{hyperref}
\usepackage[nameinlink,capitalise,noabbrev]{cleveref}
\usepackage{csquotes}

\usepackage{enumitem}
\setlist{noitemsep,topsep=3pt}
\usepackage{xcolor}

\numberwithin{equation}{section}
\setcounter{secnumdepth}{3}
\setcounter{tocdepth}{2}

\title{Sprawozdanie -- Laboratoria 2.}
\author{Jakub Kogut}
\date{\today}

\begin{document}
\maketitle

\section{Zadanie 1.}
Zadanie 1. przedstawia problem transportowy, gdzie celem jest minimalizacja kosztów przewozu towarów producentów (supplyers) do odbiorców (demand). 

\subsection{Opis modelu}
Mamy dane:
\begin{itemize}
    \item Zbiór dostawców paliwa (firmy paliwowe -- supply) $S = \{1, 2, \ldots, m\}$, gdzie $m$ to liczba dostawców.
    \item Zbiór odbiorców paliwa (lotniska -- demand) $D = \{1, 2, \ldots, n\}$, gdzie $n$ to liczba odbiorców.
    \item $s_i\geq 0$ -- ilość paliwa dostępna u dostawcy $i \in S$.
    \item $d_j \geq 0$ -- ilość paliwa wymagana przez odbiorcę $j \in D$.
    \item $c_{ij} \geq 0$ -- koszt transportu jednostki paliwa od dostawcy $i \in S$ do odbiorcy $j \in D$.
\end{itemize}
\subsubsection{Definicje zmiennych decyzyjnych}
Zmiennymi decyzyjnymi są:
\begin{itemize}
    \item $x_{ij}$ -- ilość paliwa transportowana od dostawcy $i \in S$ do odbiorcy $j \in D$.
\end{itemize}

Problem można zwizualizować grafem:
\begin{center}\begin{tikzpicture}[x=1cm,y=1.2cm]

% Kolumny
\def\xL{0}      % x lewa kolumna (dostawcy)
\def\xR{6}      % x prawa kolumna (odbiorcy)

% ======= Węzły po lewej: dostawcy S_i z podażą s_i =======
\foreach \i in {1,...,\nI} {
  \node[supplier] (S\i) at (\xL, {-(\i-1)}) {$S_{\i}$};
  \node[supply]   at ($(S\i.east)$) {$s_{\i}$};
}

% ======= Węzły po prawej: lotniska/odbiorcy D_j z popytem d_j =======
\foreach \j in {1,...,\nJ} {
  \node[demand] (D\j) at (\xR, {-(\j-1) + 0.5*(\nI-\nJ)}) {$D_{\j}$};
  \node[demandlab] at ($(D\j.west)$) {$d_{\j}$};
}

% ======= Krawędzie kompletne z opisem kosztu c_{ij} =======
\foreach \i in {1,...,\nI} {
  \foreach \j in {1,...,\nJ} {
    \draw[edge] (S\i) -- (D\j)
      node[cost] {$c_{\i\j}$};
  }
}

% ======= Opisy zbiorów =======
\node[font=\small] at (\xL, {1}) {$\text{Dostawcy } I=\{1,\dots,\nI\}$};
\node[font=\small] at (\xR, {1}) {$\text{Odbiorcy } J=\{1,\dots,\nJ\}$};
\end{tikzpicture}\end{center}

\subsubsection{Funkcja celu}
Celem jest minimalizacja całkowitych kosztów transportu paliwa, co można zapisać jako:
\[
    \min \sum_{i \in S} \sum_{j \in D} c_{ij} x_{ij}
\]
\subsubsection{Ograniczenia}
Ograniczenia, które musi spełniać model to:
\begin{itemize}
    \item podaż dla każdego dostawcy $i \in S$ nie może przekroczyć jego możliwości produkcyjnych:
        \[
            \sum_{j \in D} x_{ij} \leq s_i
        \]
    \item popyt dla każdego odbiorcy $j \in D$ musi być w pełni zaspokojony:
        \[
            \sum_{i \in S} x_{ij} = d_j
        \]
    \item zmienne decyzyjne nie mogą być ujemne:
        \[
            x_{ij} \geq 0, \quad \forall i \in S, j \in D
        \]
    \item aby egzemplarz problemu był możliwy do rozwiązania, całkowita podaż musi być mniejsza bądź równa całkowitemu popytowi:
        \[
            \sum_{i \in S} s_i \geq \sum_{j \in D} d_j
        \]
\end{itemize}

\subsection{Opis rozwiązywanych egzemplarzy}
Kod rozwiązujący egzemplarze znajduje się w pliku \texttt{kody/transport.jl}. W pliku należy podać dane: 
\begin{itemize}
    \item wektor podaży $s$,
    \item wektor popytu $d$.
    \item macierz kosztów transportu $C$,
\end{itemize}
Dla danych podanych w zadaniu mamy zadane 4 lotniska (odbiorców) i 3 firmy paliwowe (dostawców). Ilości paliwa są podane w galonach, a koszty transportu w dolarach za galon:
\begin{itemize}
    \item $s = [275000, 550000, 660000]$,
    \item $d = [110000, 220000, 330000, 440000]$,
    \item $
        C = \begin{bmatrix}
            10 &10&9&11 \\
            7&11&12&13 \\
            8&14&4&9\\
        \end{bmatrix}
        $
\end{itemize}
otrzymujemy następujące rozwiązanie:
    \[
        X = \begin{bmatrix}
            0 & 165000 & 0 & 110000 \\
            110000 & 55000 & 0 & 0 \\
            0 & 0 & 330000 & 330000 \\
        \end{bmatrix}
    \]
    gdzie $X=[x_{ij}]$ to macierz ilości paliwa transportowanego od dostawców do odbiorców. 

\subsection{Wnioski}
\begin{itemize}
    \item Całkowity, minimalny koszt transportu wynosi \num{8525000}.
    \item Tak, każda firma paliwowa wysyła paliwo do więcej niż jednego lotniska:
        \[
            \sum_{j} x_{1j} > 0, \quad \sum_{j} x_{2j} > 0, \quad \sum_{j} x_{3j} > 0
        \]
    \item Nie, nie każda firma paliwowa wysyła całkowicie swoją potencjalną produkcję paliwa:
        \[
            \sum_{j} x_{1j} = 275000 = s_1, \quad \textcolor{red}{\sum_{j} x_{2j} = 165000 < s_2}, \quad \sum_{j} x_{3j} = 660000 = s_3
        \]
\end{itemize}

\section{Zadanie 2.}


\end{document}

