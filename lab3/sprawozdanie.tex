\documentclass[11pt,a4paper]{article}

\usepackage[T1]{fontenc}
\usepackage[utf8]{inputenc}
\usepackage[polish]{babel}
\usepackage{lmodern}
\usepackage[final]{microtype}

\usepackage[a4paper,margin=2.5cm]{geometry}
\usepackage{fancyhdr}
\pagestyle{fancy}
\fancyhf{} % czyść wszystko
\lhead{Notatki z matematyki}
\rhead{\leftmark}
\cfoot{\thepage}

\usepackage{amsmath,amssymb,amsthm,mathtools}
\usepackage{siunitx} 
\sisetup{locale = DE}
\usepackage{bm}     

\usepackage[hidelinks]{hyperref}
\usepackage[nameinlink,capitalise,noabbrev]{cleveref}
\usepackage{csquotes}

\usepackage{enumitem}
\setlist{noitemsep,topsep=3pt}
\usepackage{xcolor}

\numberwithin{equation}{section}
\setcounter{secnumdepth}{3}
\setcounter{tocdepth}{2}

\title{Sprawozdanie -- Laboratoria 3.}
\author{Jakub Kogut}
\date{\today}

\begin{document}
\maketitle
\section{Wprowadzenie}
Na liście 3. mamy za zadanie przeprowadzić implementację i analizę algorytmów znajdowania najkrótszych ścieżek w grafach. Za dane mamy użyć grafy wygenerowane przez generator oraz sieć drogową USA napisane przez \texttt{DIMACS}.

\section{Implementacja}
Wszystkie zadania zaimplementowałem w języku \texttt{C++}. Kody algorytmów znajdują się w \texttt{algo.cpp}, natomiast instrukcje uruchamiania w \texttt{README.md}. Wszytkie programy kożystają wyłącznie z biblioteki standardowej \texttt{C++}, można je kompilować przy użyciu \texttt{make}
\subsection{Algorytm \texttt{DIJKSTRY}}
Algorytm Dijkstry zaimplementowałem przy użyciu kolejki priorytetowej \texttt{std::priority\_queue} z biblioteki standardowej \texttt{C++}. Jej złożoność na operacje \texttt{push} i \texttt{pop} wynosi \(O(\log n)\), gdzie \(n\) to liczba elementów w kolejce. W najgorszym przypadku algorytm Dijkstry odwiedza wszystkie wierzchołki i krawędzie grafu, co daje łączną złożoność czasową \(O((V + E) \log V)\), gdzie \(V\) to liczba wierzchołków, a \(E\) to liczba krawędzi w grafie.

\subsection{Algorytm \texttt{DIALA}}

\subsection{Algorytm \texttt{RADIX-HEAP}}

\section{Analiza rodzin grafów}

\section{Wyniki, wnioski}

\section{Podsumowanie}

\end{document}

